\documentclass[12pt,a4paper]{article}
\usepackage[utf8]{inputenc}
\usepackage{polski}
\usepackage{graphicx}

\newcommand{\screenshot}[1]{\\\begin{minipage}[c]{\textwidth}\hspace{0em}\includegraphics[width=\textwidth]{#1}\end{minipage}\vspace{0em}}
\newcommand{\json}[1]{\texttt{#1?format=json}}
\newcommand{\soap}[1]{\texttt{#1?format=soap}}

\title{Aplikacja \emph{Polls}}
\author{Grzegorz Kowalski\\\texttt{daneos@daneos.com}}
\date{\today}

\begin{document}
\maketitle

\begin{abstract}
\emph{Polls} to aplikacja webowa stworzona przy użyciu frameworku Django. Udostępnia prosty system ankiet wraz z panelem administracyjnym.\\
Ankiety identyfikowane są przez automatycznie przypisywany numer. Dostępna jest również lista wszystkich ankiet, oraz widoczne dla użytkowników wyniki ankiet w formacie procentowym.\\
Panel administracyjny udostępnia opcje dodawania, usuwania i modyfikacji ankiet.\\
Każda akcja może być wykonana za pomocą interfejsu webowego, zapytań JSON lub SOAP, należy wówczas przekazać parametr odpowiednio \json{} lub \soap{}.
\end{abstract}

\section{Lista ankiet}
Po wejściu na adres URL aplikacji otrzymujemy widok dostępnych w systemie ankiet, posortowanych według daty dodania, zaczynając od najnowszych.
\screenshot{poll_list.png}
Kliknięcie na tytuł ankiety przenosi nas do widoku pojedyńczej ankiety.
\newpage
Wchodząc na \json{/polls/} otrzymujemy następującą odpowiedź:
\begin{verbatim}
[
  {
    "date": "2015-06-11 17:52:10+00:00",
    "question": "Czy jutro wstanie s\u0142o\u0144ce?",
    "id": 2
  },
  {
    "date": "2015-06-11 17:12:08+00:00",
    "question": "Co dzi\u015b na obiad?",
    "id": 1
  }
]
\end{verbatim}
Z kolei \soap{/polls/} zwraca:
\begin{verbatim}
<?xml version="1.0" encoding="UTF-8"?>
<SOAP-ENV:Envelope
  SOAP-ENV:encodingStyle="http://schemas.xmlsoap.org/soap/encoding/"
  xmlns:SOAP-ENC="http://schemas.xmlsoap.org/soap/encoding/"
  xmlns:xsi="http://www.w3.org/1999/XMLSchema-instance"
  xmlns:SOAP-ENV="http://schemas.xmlsoap.org/soap/envelope/"
  xmlns:xsd="http://www.w3.org/1999/XMLSchema"
>
  <SOAP-ENV:Body>
    <v1 SOAP-ENC:arrayType="ns1:SOAPStruct[2]" xsi:type="SOAP-ENC:Array" xmlns:ns1="http://soapinterop.org/xsd" SOAP-ENC:root="1">
      <item>
        <date xsi:type="xsd:string">2015-06-11 17:52:10+00:00</date>
        <question xsi:type="xsd:string">Czy jutro wstanie słońce?</question>
        <id xsi:type="xsd:int">2</id>
      </item>
      <item>
        <date xsi:type="xsd:string">2015-06-11 17:12:08+00:00</date>
        <question xsi:type="xsd:string">Co dziś na obiad?</question>
        <id xsi:type="xsd:int">1</id>
      </item>
    </v1>
  </SOAP-ENV:Body>
</SOAP-ENV:Envelope>
\end{verbatim}

\section{Ankieta}
Widok ankiety pokazuje pytanie oraz dostępne odpowiedzi w formie formularza do głosowania.
\screenshot{poll.png}
Kliknięcie przycisku \emph{Vote!} powoduje zapisanie zaznaczonej opcji, oraz przejście do wyników głosowania. Jeśli żadna opcja nie była zaznaczona, wyświetlony zostanie komunikat o błędzie. Do wyników ankiety można również przejść klikając link \emph{View results} w prawym dolnym rogu ankiety.\\
Adres URL widoku ankiety to \texttt{/polls/\emph{<id>}/}.\\
Wejście na adres np. \json{/polls/1/} zwróci przykładową ankietę:
\begin{verbatim}
{
  "date": "2015-06-11 17:12:08+00:00",
  "question": "Co dzi\u015b na obiad?",
  "id": 1,
  "choices": [
    {
      "id": 1,
      "choice": "Bu\u0142ka z chlebem"
    },
    {
      "id": 2,
      "choice": "Woda na mokro"
    },
    {
      "id": 3,
      "choice": "Chleb posmarowany no\u017cem"
    },
    {
      "id": 4,
      "choice": "Nic"
    }
  ]
}
\end{verbatim}
Głosowanie za pomocą JSON polega na wysłaniu zapytania POST z na adres \json{/polls/\emph{<id>}/vote/} w treści którego znajduje się obiekt JSON w postaci
\texttt{\{"choice": \emph{<id\_wybranej\_opcji>}\}}, np.
\begin{verbatim}
POST /polls/1/vote/?format=json HTTP/1.1
Host: localhost
Content-Type: application/json
Content-Length: 13

{"choice": 1}
\end{verbatim}
Odpowiedzią serwera jest \texttt{302 Found} z nagłówkiem \texttt{Location} wskazującym na wyniki ankiety, dla przykładowego requesta:
\begin{verbatim}
HTTP/1.0 302 FOUND
Date: Fri, 19 Jun 2015 16:41:16 GMT
Server: WSGIServer/0.1 Python/2.7.6
X-Frame-Options: SAMEORIGIN
Content-Type: text/html; charset=utf-8
Location: http://localhost/polls/1/results/
\end{verbatim}
\vspace{3em}
Pobieranie ankiety w formacie SOAP: \soap{/polls/\emph{<id>}/}
\begin{verbatim}
<?xml version="1.0" encoding="UTF-8"?>
<SOAP-ENV:Envelope
  SOAP-ENV:encodingStyle="http://schemas.xmlsoap.org/soap/encoding/"
  xmlns:SOAP-ENC="http://schemas.xmlsoap.org/soap/encoding/"
  xmlns:xsi="http://www.w3.org/1999/XMLSchema-instance"
  xmlns:SOAP-ENV="http://schemas.xmlsoap.org/soap/envelope/"
  xmlns:xsd="http://www.w3.org/1999/XMLSchema"
>
  <SOAP-ENV:Body>
    <v1 SOAP-ENC:root="1">
      <date xsi:type="xsd:string">2015-06-11 17:12:08+00:00</date>
      <question xsi:type="xsd:string">Co dziś na obiad?</question>
      <id xsi:type="xsd:int">1</id>
      <choices SOAP-ENC:arrayType="ns1:SOAPStruct[4]" xsi:type="SOAP-ENC:Array" xmlns:ns1="http://soapinterop.org/xsd">
        <item>
          <id xsi:type="xsd:int">1</id>
          <choice xsi:type="xsd:string">Bułka z chlebem</choice>
        </item>
        <item>
          <id xsi:type="xsd:int">2</id>
          <choice xsi:type="xsd:string">Woda na mokro</choice>
        </item>
        <item>
          <id xsi:type="xsd:int">3</id>
          <choice xsi:type="xsd:string">Chleb posmarowany nożem</choice>
        </item>
        <item>
          <id xsi:type="xsd:int">4</id>
          <choice xsi:type="xsd:string">Nic</choice>
        </item>
      </choices>
    </v1>
  </SOAP-ENV:Body>
</SOAP-ENV:Envelope>
\end{verbatim}
Głosowanie za pomocą SOAP odbywa się podobnie jak w przypdaku JSON. Przykład:
\begin{verbatim}
POST /polls/1/vote/?format=soap HTTP/1.1
Host: localhost
Content-Type: application/soap+xml
Content-Length: 479

<?xml version="1.0" encoding="UTF-8"?>
<SOAP-ENV:Envelope
  SOAP-ENV:encodingStyle="http://schemas.xmlsoap.org/soap/encoding/"
  xmlns:SOAP-ENC="http://schemas.xmlsoap.org/soap/encoding/"
  xmlns:xsi="http://www.w3.org/1999/XMLSchema-instance"
  xmlns:SOAP-ENV="http://schemas.xmlsoap.org/soap/envelope/"
  xmlns:xsd="http://www.w3.org/1999/XMLSchema"
>
  <SOAP-ENV:Body>
    <v1 SOAP-ENC:root="1">
      <choice xsi:type="xsd:int">3</choice>
    </v1>
  </SOAP-ENV:Body>
</SOAP-ENV:Envelope>
\end{verbatim}
Tak samo jak w przypadku JSON, serwer odpowiada \texttt{302 Found} z odpowiednio ustawionym nagłówkiem \texttt{Location}.

\section{Wyniki}
Widok wyników prezentuje procentowy udział odpowiedzi w całości głosów.
\screenshot{poll_results.png}
W prawym dolnym rogu widoczna jest całkowita ilość głosów oddanych w ankiecie.
Widok JSON: \json{/polls/\emph{<id>}/results/}
\begin{verbatim}
{
  "date": "2015-06-11 17:12:08+00:00",
  "total_votes": 4,
  "question": "Co dzi\u015b na obiad?",
  "id": 1,
  "choices": [
    {
      "votes": 0,
      "id": 1,
      "choice": "Bu\u0142ka z chlebem"
    },
    {
      "votes": 2,
      "id": 2,
      "choice": "Woda na mokro"
    },
    {
      "votes": 1,
      "id": 3,
      "choice": "Chleb posmarowany no\u017cem"
    },
    {
      "votes": 1,
      "id": 4,
      "choice": Nic"
    }
  ]
}
\end{verbatim}
Widok SOAP: \soap{/polls/\emph{<id>}/results/}
\begin{verbatim}
<?xml version="1.0" encoding="UTF-8"?>
<SOAP-ENV:Envelope
  SOAP-ENV:encodingStyle="http://schemas.xmlsoap.org/soap/encoding/"
  xmlns:SOAP-ENC="http://schemas.xmlsoap.org/soap/encoding/"
  xmlns:xsi="http://www.w3.org/1999/XMLSchema-instance"
  xmlns:SOAP-ENV="http://schemas.xmlsoap.org/soap/envelope/"
  xmlns:xsd="http://www.w3.org/1999/XMLSchema"
>
  <SOAP-ENV:Body>
    <v1 SOAP-ENC:root="1">
      <date xsi:type="xsd:string">2015-06-11 17:12:08+00:00</date>
      <total_votes xsi:type="xsd:int">4</total_votes>
      <question xsi:type="xsd:string">Co dziś na obiad?</question>
      <id xsi:type="xsd:int">1</id>
      <choices SOAP-ENC:arrayType="ns1:SOAPStruct[4]" xsi:type="SOAP-ENC:Array" xmlns:ns1="http://soapinterop.org/xsd">
        <item>
          <votes xsi:type="xsd:int">0</votes>
          <id xsi:type="xsd:int">1</id>
          <choice xsi:type="xsd:string">Bułka z chlebem</choice>
        </item>
        <item>
          <votes xsi:type="xsd:int">2</votes>
          <id xsi:type="xsd:int">2</id>
          <choice xsi:type="xsd:string">Woda na mokro</choice>
        </item>
        <item>
          <votes xsi:type="xsd:int">1</votes>
          <id xsi:type="xsd:int">3</id>
          <choice xsi:type="xsd:string">Chleb posmarowany nożem</choice>
        </item>
        <item>
          <votes xsi:type="xsd:int">1</votes>
          <id xsi:type="xsd:int">4</id>
          <choice xsi:type="xsd:string">Nic</choice>
        </item>
      </choices>
    </v1>
  </SOAP-ENV:Body>
</SOAP-ENV:Envelope>
\end{verbatim}
Jak widać widok wyników od widoku ankiety rózni się tylko tym, że podana jest ilość wszystkich głosów i ilość głosów oddana na każdą opcję.

\newpage
\section{Panel administracyjny}
Automatycznie generowany panel administracyjny pozwala na dodawanie nowych ankiet oraz modyfikację lub usuwanie istniejących.
\screenshot{admin.png}
Do każdej ankiety można dodać dowolną ilość odpowiedzi.\\
Ten element nie ma wersji JSON ani SOAP.

\end{document}