\documentclass[12pt,a4paper]{article}
\usepackage[utf8]{inputenc}
\usepackage{polski}
\usepackage{graphicx}

\newcommand{\screenshot}[1]{\\\begin{minipage}[c]{\textwidth}\hspace{0em}\includegraphics[width=\textwidth]{#1}\end{minipage}\vspace{0em}}

\title{Aplikacja \emph{Polls}}
\author{Grzegorz Kowalski\\\texttt{daneos@daneos.com}}
\date{\today}

\begin{document}
\maketitle

\begin{abstract}
\emph{Polls} to aplikacja webowa stworzona przy użyciu frameworku Django. Udostępnia prosty system ankiet wraz z panelem administracyjnym.\\
Ankiety identyfikowane są przez automatycznie przypisywany numer. Dostępna jest również lista wszystkich ankiet, oraz widoczne dla użytkowników wyniki ankiet w formacie procentowym.\\
Panel administracyjny udostępnia opcje dodawania, usuwania i modyfikacji ankiet.
\end{abstract}

\section{Lista ankiet}
Po wejściu na adres URL aplikacji otrzymujemy widok dostępnych w systemie ankiet, posortowanych według daty dodania, zaczynając od najnowszych.
\screenshot{poll_list.png}
Kliknięcie na tytuł ankiety przenosi nas do widoku pojedyńczej ankiety.

\newpage
\section{Ankieta}
Widok ankiety pokazuje pytanie oraz dostępne odpowiedzi w formie formularza do głosowania.
\screenshot{poll.png}
Kliknięcie przycisku \emph{Vote!} powoduje zapisanie zaznaczonej opcji, oraz przejście do wyników głosowania. Jeśli żadna opcja nie była zaznaczona, wyświetlony zostanie komunikat o błędzie. Do wyników ankiety można również przejść klikając link \emph{View results} w prawym dolnym rogu ankiety.

\section{Wyniki}
Widok wyników prezentuje procentowy udział odpowiedzi w całości głosów.
\screenshot{poll_results.png}
W prawym dolnym rogu widoczna jest całkowita ilość głosów oddanych w ankiecie.

\newpage
\section{Panel administracyjny}
Automatycznie generowany panel administracyjny pozwala na dodawanie nowych ankiet oraz modyfikację lub usuwanie istniejących.
\screenshot{admin.png}
Do każdej ankiety można dodać dowolną ilość odpowiedzi.

\end{document}