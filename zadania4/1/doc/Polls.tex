\documentclass[12pt,a4paper]{article}
\usepackage[utf8]{inputenc}
\usepackage{polski}
\usepackage{graphicx}
\usepackage{listingsutf8}

\lstset{
  basicstyle=\tiny,
  frame=single,
  showstringspaces=false
}

\newcommand{\screenshot}[1]{\\\begin{minipage}[c]{\textwidth}\hspace{0em}\includegraphics[width=\textwidth]{#1}\end{minipage}\vspace{0em}}
\newcommand{\adres}[1]{\emph{http://polls-989.appspot.com/#1}}

\title{Aplikacja \emph{Polls}}
\author{Grzegorz Kowalski\\\texttt{daneos@daneos.com}}
\date{\today}

\begin{document}
\maketitle

\begin{abstract}
\emph{Polls} to aplikacja webowa stworzona przy użyciu Google App Engine. Obsługuje dodawanie i modyfikację ankiet przez interfejs webowy i REST API.\\
Adres aplikacji to \adres{}, REST API dostępne jest pod adresem \adres{rest/poll}.
\end{abstract}

\section{Tworzenie ankiety}
Wchodząc na \adres{create} mamy prosty formularz służący do dodawania ankiet.
\screenshot{new_poll.png}
Po zapisaniu danych aplikacja wyświetla link do nowo utworzonej ankiety.
\screenshot{created.png}
\newpage
Aby dodać ankietę przez REST, należy wysłać następujący (przykładowy) request:
\lstinputlisting{new_poll_req.json}
Serwer odpowiada wysyłając utworzoną ankietę.

\section{Lista ankiet}
Po wejściu na adres URL aplikacji otrzymujemy widok dostępnych w systemie ankiet, posortowanych według daty dodania, zaczynając od najnowszych.
\screenshot{poll_list.png}
Kliknięcie na tytuł ankiety przenosi nas do widoku pojedyńczej ankiety.
Wchodząc na adres REST API również dostajemy listę ankiet:
\lstinputlisting{rest_poll_list.json}

\section{Ankieta}
Widok ankiety pokazuje pytanie oraz dostępne odpowiedzi w formie formularza do głosowania.
\screenshot{poll.png}
Kliknięcie przycisku \emph{Vote!} powoduje zapisanie zaznaczonej opcji, oraz przejście do wyników głosowania. Jeśli żadna opcja nie była zaznaczona, wyświetlony zostanie komunikat o błędzie. Do wyników ankiety można również przejść klikając link \emph{View results} w prawym dolnym rogu ankiety.\\
Adres URL widoku ankiety to \adres{poll/<id>}\\
Widok RESTowy ankiety:
\lstinputlisting{rest_poll.json}
Głosowanie za pomocą REST API polega na zaktualizowaniu wpisów \emph{choices} w ankiecie za pomocą HTTP PUT.
\lstinputlisting{update_poll_req.json}

\section{Wyniki}
Widok wyników prezentuje ilość głosów na każdą z opcji.
\screenshot{poll_results.png}
W prawym dolnym rogu widoczna jest całkowita ilość głosów oddanych w ankiecie.
W REST API za widok wyników służy widok ankiety, ponieważ przesyła wszystkie niezbędne informacje.

\end{document}