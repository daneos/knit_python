\documentclass[12pt,a4paper]{article}
\usepackage[utf8]{inputenc}
\usepackage{polski}
\usepackage{graphicx}

\newcommand{\screenshot}[1]{\\\begin{minipage}[c]{\textwidth}\hspace{0em}\includegraphics[width=\textwidth]{#1}\end{minipage}\vspace{0em}}
\newcommand{\adres}[1]{\emph{http://potent-clarity-98922.appspot.com/\_ah/api/#1}}

\title{Aplikacja \emph{Polls}}
\author{Grzegorz Kowalski\\\texttt{daneos@daneos.com}}
\date{\today}

\begin{document}
\maketitle

\begin{abstract}
\emph{Polls} to aplikacja obsługująca ankiety, stworzona przy użyciu Google Cloud Endpoints.\\
Adres aplikacji to \adres{}.
\end{abstract}

\section{API Explorer}
Google Cloud udostępnia API Explorer, dostępny pod adresem \adres{explorer}.
\screenshot{api_list.png}

\section{Tworzenie ankiety}
Używając funkcji \emph{polls.polls.create} możemy utworzyć nową ankietę. Nie trzeba uzupełniać wszystkich pól, ponieważ większość (np.~klucze, data, ilość głosów) uzupełniane są automatycznie przez aplikację.\\
Tworzenie odbywa się poprzez request POST.
\screenshot{new_poll.png}
Funkcja odpowiada wysyłając nowo utworzoną ankietę

\section{Lista ankiet}
Funkcja \emph{polls.polls.list} służy do pobrania listy dostępnych w systemie ankiet.
\screenshot{poll_list.png}

\section{Ankieta}
Funkcja \emph{polls.polls.get} pobiera pojedyńczą ankietę na podstawie podanego ID.
\screenshot{poll.png}

\section{Głosowanie}
Funkcja \emph{polls.polls.vote} służy do głosowania w ankiecie. Przyjmuje ID ankiety oraz ID wybranej odpowiedzi.
Jest to realizowane za pomocą requestu PUT na adres ankiety.
\screenshot{vote.png}
Funkcja odpowiada wysyłając zaktualizowaną ankietę.

\end{document}