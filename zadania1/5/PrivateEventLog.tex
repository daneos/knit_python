\documentclass[12pt,a4paper]{article}
\usepackage[utf8]{inputenc}
\usepackage{polski}

\title{Private Event Log}
\author{Grzegorz Kowalski\\\texttt{daneos@daneos.com}}
\date{\today}

\begin{document}
\maketitle

\begin{abstract}
Private Event Log to usługa REST zbudowana za pomocą frameworku Django.\\
Pozwala na dodawanie wpisów za pomocą zapytania HTTP POST oraz pobranie zapisanych wpisów za pomocą HTTP GET.\\
Private Event Log pozwala na wyszukiwanie wpisów na podstawie kategorii (\#), osób (@) oraz czasu dodania.
\end{abstract}

\section{Dodawanie wpisów}
Wpisy dodawane są do logu za pomocą zapytania POST na adres /.
Aplikacja przyjmuje ciąg znaków sformatowany jako JSON.\\
Przykładowe zapytanie:
\begin{verbatim}
POST / HTTP/1.1
Host: example.com
Content-type: application/json
Content-length: 35

"I just won a lottery #update @all"
\end{verbatim}
Efektem wykonania tego zapytania będzie odpowiedź serwera 201 Created.\\
Aplikacja utworzy nowy wpis w kategorii \texttt{update} przypisany do osób \texttt{all} (w domyśle wszystkich).
Możliwe jest przypisanie wpisu do kilku kategorii lub kilku osób na raz, używając kilku bloków \texttt{\#\emph{kategoria}} i/lub \texttt{@\emph{osoba}}.\\
Private Event Log \textbf{nie} usuwa tagów kategorii i osób z tekstu, ponieważ mogą być integralną częścią wiadomości (np.~\texttt{I just lost lottery prize at \#poker with @jane}).

\section{Pobieranie wpisów}
Wpisy pobiera się za pomocą zapytań GET.\\
Zapytanie na adres / pobiera 10 ostatnio dodanych wpisów ze wszystkich kategorii i osób.
Aby pobrać ostatnie wpisy z konkretnej kategorii należy skierować zapytanie na adres \texttt{/category/\emph{kategoria}}, a aby pobrać wpisy przypisane do osób -- na \texttt{/person/\emph{osoba}}. Istnieje również możliwość pobrania wpisu dodanego w konkretnym momencie, poprzez zapytanie na adres \texttt{/time/\emph{unix-timestamp}}.\\
Wszystkie zapytania zwracają maksymalnie 10 ostatnich pasujących wpisów w formacie JSON.

\end{document}