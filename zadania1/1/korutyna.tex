\documentclass[12pt,a4paper]{article}
\usepackage[utf8]{inputenc}
\usepackage{polski}

\begin{document}

\title{Korutyny}
\author{Grzegorz Kowalski\\\texttt{daneos@daneos.com}}
\date{\today}
\maketitle

\section{Opis}
\textbf{Korutyna} (jest takie słowo w języku polskim?) jest rodzajem podprogramu, którego cechą charakterystyczną jest możliwość wielokrotnego przekazania sterowania do innej korutyny i kontynuowania wykonania od miejsca, w którym zostało zawieszone.\\
W językach funkcyjnych odpowiednikiem korutyny jest kontynuacja, czyli obiekt reprezentujący zachowany stan wykonywania programu.\\
Koncepcyjnie korutyny są podobne do wątków i w wielu zastosowaniach są ich dobrym przybliżeniem, jednak podstawową różnicą jest brak współbieżności korutyn.
Współdziałaniem wątków zarządza system operacyjny, natomiast korutyn - środowisko języka programowania (interpreter, biblioteka standardowa), chociaż niektóre języki programowania (np. Lua) pozwalają na implementacje korutyn za pomocą wątków systemowych.

\section{Przykład}
\begin{verbatim}
coroutine producent                   coroutine konsument
{                                     {
    while(!kolejka.full())                while(!kolejka.empty())
    {                                     {
        kolejka += produkuj()                 konsumuj(kolejka)
        yield konsument                       yield producent
    }                                     }
}                                     }
\end{verbatim}
Powyższy pseudokod tworzy producenta, który wypełnia kolejkę i wywołuje kosumenta, który zabiera elementy z kolejki. Kolejne wywołania przeplatają się, kontynuując wykonanie w miejscu w którym skończyły.

\end{document}